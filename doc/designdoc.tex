%%%%%%%%%%%%%%%%%%%%%%%%%%%%%%%%%%%%%%%%%
% University/School Laboratory Report
% LaTeX Template
% Version 3.1 (25/3/14)
%
% This template has been downloaded from:
% http://www.LaTeXTemplates.com
%
% Original author:
% Linux and Unix Users Group at Virginia Tech Wiki
% (https://vtluug.org/wiki/Example_LaTeX_chem_lab_report)
%
% License:
% CC BY-NC-SA 3.0 (http://creativecommons.org/licenses/by-nc-sa/3.0/)
%
%%%%%%%%%%%%%%%%%%%%%%%%%%%%%%%%%%%%%%%%%

%----------------------------------------------------------------------------------------
%	PACKAGES AND DOCUMENT CONFIGURATIONS
%----------------------------------------------------------------------------------------

\documentclass{article}

\usepackage{siunitx} % Provides the \SI{}{} and \si{} command for typesetting SI units
\usepackage{graphicx} % Required for the inclusion of images
\usepackage{natbib} % Required to change bibliography style to APA
\usepackage{amsmath} % Required for some math elements
\usepackage{cnn, comment}
\usepackage{subfigure}
\usepackage{standalone}

\setlength\parindent{0pt} % Removes all indentation from paragraphs

\renewcommand{\labelenumi}{\alph{enumi}.} % Make numbering in the enumerate environment by letter rather than number (e.g. section 6)

%\usepackage{times} % Uncomment to use the Times New Roman font

%----------------------------------------------------------------------------------------
%	DOCUMENT INFORMATION
%----------------------------------------------------------------------------------------

\title{Deep RL Stock Trading} % Title

\author{Scott C. Waggener} % Author name

\date{\today} % Date for the report

\begin{document}

\maketitle % Insert the title, author and date

%----------------------------------------------------------------------------------------
\section{Objective}
%----------------------------------------------------------------------------------------


To trade stocks (and eventually other securities) with a capability
that matches or exceeds human traders.

%----------------------------------------------------------------------------------------
\section{Implementation}
%----------------------------------------------------------------------------------------

In order to effectively implement the required functionality for this
project in a modular and easily maintainable/modifiable way, the
following approach is suggested

\subsection{Gym}

	An environment for training and evaluating reinforcement learning
	algorithms on financial data must be created. When run, the operator
	will specify

	\begin{enumerate}
		\item A timespan over which historical prices will be evaluated
		\item A trained model that will receive price data iteratively
		\item An initial portfolio value, possibly with constraints on max
			loss
	\end{enumerate}

	From these inputs, the training environment should do the following

	\begin{enumerate}
		\item Seed an initial window of price data that ends at the
			opening price of stocks on the start day to the agent. This
			should be achieved by explicitly creating a window, rather than
			returning copied values. The output will be a
			$10,000 \times 180 \times 5$ tensor assuming $10,000$ securities
			over a $180$ day window with $5$ price metrics.
		\item The agent should yield a key value structure representing
			stocks to buy, sell, or short. Assume that selling a stock that
			isn't currently owned indicates a shorting position.
		\item The environment steps to the next trading day and yields a
			new portfolio value, along with a new window of price data.
	\end{enumerate}

%----------------------------------------------------------------------------------------
\section{Architecture}
%----------------------------------------------------------------------------------------

\subsection{Time Insensitive Analysis}

Some technical analysis metrics used by human traders are relatively
time insensitive, meaning that they provide insights into the general
viability of a security by looking at trends over a long period of
time. For example, support and resistance levels can be defined over
some period of relevance, call it $180$ days. The times at which these
levels were established are not of massive signficance.
\newline

As such, there may be value in a network architecture that exhibits
temporal translation invariance. The approach will be to apply a
convolutional nerual network to price and volume data over a
constrained historical window.

\begin{figure}[h]
	\centering
	\includestandalone{cnn}
	\caption{Possible convolutional architecture.}
	\label{fig:cnn}
\end{figure}

\end{document}

